\documentclass[a4paper, titlepage]{article}

% For equations
\usepackage{amsmath}

% For including figures
\usepackage{graphicx}
\usepackage{float}

% Bibiliography setup
\usepackage[square]{natbib}
\bibliographystyle{agsm}
\usepackage[nottoc]{tocbibind}

% For typesetting matlab
\usepackage{listings}
\usepackage{color} %red, green, blue, yellow, cyan, magenta, black, white
\definecolor{mygreen}{RGB}{28,172,0} % color values Red, Green, Blue
\definecolor{mylilas}{RGB}{170,55,241}

\lstset{language=Matlab,%
    basicstyle=\small,
    breaklines=true,%
    frame = single,
    morekeywords={matlab2tikz},
    keywordstyle=\color{blue},%
    morekeywords=[2]{1}, keywordstyle=[2]{\color{black}},
    identifierstyle=\color{black},%
    stringstyle=\color{mylilas},
    commentstyle=\color{mygreen},%
    showstringspaces=false,
    numbers=left,%
    numberstyle={\tiny \color{black}},% size of the numbers
    numbersep=9pt, % this defines how far the numbers are from the text
    emph=[1]{for,end,break},emphstyle=[1]\color{red}, %some words to emphasise
    %emph=[2]{word1,word2}, emphstyle=[2]{style},    
}


%\title{Laboratory Work 1\\
%Coupled Drives\\
%\large EEA004}
%\author{Dan Thilderkvist, Philip Gutierrez}

\begin{document}

%\maketitle

\begin{titlepage}
  \begin{center}
    \vspace*{1cm}
    \includegraphics[scale=1.0]{../figures/hig_logo_eng.png}\\
    \vspace{1.5cm}
    \large EEA004 - Multivariable and Nonlinear Control Systems\\
    \large Laboratory Work 1\\
    \vspace{1.5cm}
    Group 4\\
    Dan Thilderkvist and Philip Gutierrez\\
    dan.thilderkvist@hotmail.com philipgutierrez67@gmail.com\\
    Files: TBD\\
    \vspace{1cm}
    \today
  \end{center}
\end{titlepage}

\section{Introduction}
In this experiment a coupled system is decoupled and controlled.
The system studied is the control of speed and tension of a material in a continuous system.
The experiments herein is carried out using Simulink with a transfer function model based on a real experimental setup, figure \ref{fig:expSys}.

\begin{figure}[h!]
\center
\includegraphics[scale=0.65]{../figures/experimentSystem.png}
\caption{Experimental setup that is mimicked in Simulink.}
\label{fig:expSys}
\end{figure}

The system inputs are the motor torques $u_1, u_2$ and the system outputs are the jockey pulleys speed $y_1$ and arm deflection $y_2$.
The transfer function representing this system in Simulink is:

\begin{equation}
G(s) = 
\begin{pmatrix}
\frac{0.87}{2(0.66s + 1)} & \frac{0.87}{2(0.66s + 1)} \\[6pt]
-\frac{272}{s^2 + 8.05s + 263} & \frac{272}{s^2 + 8.05s + 263}
\end{pmatrix}
\label{equ:transFunc}
\end{equation}

Throughout this report the behavior of this system will be studied, both the coupled system (\ref{equ:transFunc}) and the decoupled system that will be developed.

\section{Study the step response without decoupling}
In this first experiment the system is stimulated to showcase the coupling.
Motor 1 $u_1$ is set constant $1$ while Motor 2 $u_2$ is pulsed with square wave of period 50s ans amplitude 1.
The pulse generator start high and the resulting speed and tension of the material can be seen in figure X.



The resulting speed and tension both clearly show a repeating pattern of period 50s.
This indicate that there is coupling from Motor 2 to both outputs.
While Motor 2 is running, both motors run at the same speed and there is no tension.
If Motor 2 is stopped (with positive direction clockwise), the tension on the material increase between Motor 1 and Motor 2 due to either the material slipping on Motor 2 or the material having to sustain the load of rotating motor 2.
The tension for the remainder of the loop will then decrease (between the jockey and the motors) and this is the measured tension output.

\section{Study the step response for a decoupled system}
To decouple the system a decoupling pre-compensator can be added the system.
This will take the input $u_1, u_2$ and output a modified signal $v_1, v_2$ that will be the new input to the coupled system (\ref{equ:transFunc}).
The transfer function for the pre-compensator is:

\begin{equation}
W(s) = 
\begin{pmatrix}
1 & -1 \\ 1 & 1
\end{pmatrix}
\end{equation}

Calculating the new transfer function from input $U(s)$ to output $Y(s)$ result in:

\begin{equation}
\begin{split}
\tilde{G}(s) = G(s)W(s) &= 
\begin{pmatrix}
\frac{0.87}{2(0.66s + 1)} & \frac{0.87}{2(0.66s + 1)} \\[6pt]
-\frac{272}{s^2 + 8.05s + 263} & \frac{272}{s^2 + 8.05s + 263}
\end{pmatrix}
\begin{pmatrix}
1 & -1 \\ 1 & 1
\end{pmatrix} = \\
&= \begin{pmatrix}
2\frac{0.87}{2(0.66s + 1)} & 0 \\[6pt]
0 & 2\frac{272}{s^2 + 8.05s + 263}
\end{pmatrix} = \\
&= \begin{pmatrix}
\frac{0.87}{0.66s + 1} & 0 \\[6pt]
0 & \frac{544}{s^2 + 8.05s + 263}
\end{pmatrix}
\end{split}
\label{equ:decoupled}
\end{equation}

With only elements on the diagonal, the system is now completely decoupled.
Rerunning the experiment were one input is constant ($u_1$) and the other is pulsed ($u_2$) will generate the results in figure X.
If the inputs were to be exchanged ($u_1$ pulsed, $u_2$ constant) the result is as in figure X.
Note now that the inputs $u_1$ and $u_2$ are no longer the motor torques as in the previous section but instead the inputs to the pre-compensator.
The motor torque is the output of the pre-compensator.

With the pre-compensator, the experiment just carried out show that $u_1$ can be used to control the material speed $y_1$ and $u_2$ can be used to control the material tension $y_2$ over the jockey pulley.
This could also have been directly read out from the new transfer function $\tilde{G}(s)$ as the elements are on the diagonal.

\section{Study the decoupled system with feedback control}



\clearpage
\bibliography{reference}

\clearpage
\appendix

\section{Example Section}
This is an example reference \citep{glad00}.

\begin{figure}[h!]
\center
%\includegraphics[scale=0.8]{../code/figures/exampleFigure.png}
\caption{Example caption.}
\label{fig:exampleLable}
\end{figure}

\lstinputlisting{../code/main.m}

\end{document}