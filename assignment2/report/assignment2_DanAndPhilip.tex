\documentclass[a4paper, titlepage]{article}

% For equations
\usepackage{amsmath}

% For including figures
\usepackage{graphicx}
\usepackage{float}

% Bibiliography setup
\usepackage[square]{natbib}
\bibliographystyle{agsm}
\usepackage[nottoc]{tocbibind}

% For typesetting matlab
\usepackage{listings}
\usepackage{color} %red, green, blue, yellow, cyan, magenta, black, white
\definecolor{mygreen}{RGB}{28,172,0} % color values Red, Green, Blue
\definecolor{mylilas}{RGB}{170,55,241}

\lstset{language=Matlab,%
    basicstyle=\small,
    breaklines=true,%
    frame = single,
    morekeywords={matlab2tikz},
    keywordstyle=\color{blue},%
    morekeywords=[2]{1}, keywordstyle=[2]{\color{black}},
    identifierstyle=\color{black},%
    stringstyle=\color{mylilas},
    commentstyle=\color{mygreen},%
    showstringspaces=false,
    numbers=left,%
    numberstyle={\tiny \color{black}},% size of the numbers
    numbersep=9pt, % this defines how far the numbers are from the text
    emph=[1]{for,end,break},emphstyle=[1]\color{red}, %some words to emphasise
    %emph=[2]{word1,word2}, emphstyle=[2]{style},    
}


\title{Assignment 2\\
System description and analysis\\
\large EEA004}
\author{Dan Thilderkvist, Philip Gutierrez}

\begin{document}

\maketitle

\section{Introduction}
This assignment is a continuation from a previous assignment where the dynamics of an air handling unit consisting of a heater and humidifier was analyzed.
In this report the air handling unit will be analyzed and controlled more in-depth.
That entails looking at more general methods for choosing input output control pairs, constructing decoupling matrices and using optimal state feedback to control the system.
The air handling system was modeled in the previous assignment but the plant transfer function is stated here for completeness:

\begin{equation}
G(s) = 
\begin{pmatrix}
\frac{15}{50s + 1} && \frac{-11}{10s + 1} \\[6pt]
\frac{-25}{50s + 1} && \frac{70}{10s + 1}
\end{pmatrix}
\end{equation}

\subsection{Theory}

A measurement of system input output interaction is the Relative Gain Array (RGA).
The RGA for any matrix is defined as below \citep[~p.219]{glad00}:

\begin{equation}
RGA(A) = A.*(A^{-1})^T
\end{equation}

Here the multiplication is element-wise and the inverse can be exchanged for the pseudo-inverse if $A$ is not square.
For a system to be controlled by decoupled controllers the input-output pairs should be chosen such that the RGA matrix elements are non-negative for $s=0$ if A is a transfer function $G(s)$.

TODO: Part about decoupling matrices

TODO: Part about LQR/LQG control

\section{Method}
This section describes the methodology used to obtain the requested results of the assignment.
It will go through them step by step, starting by choosing a input-output pairing.

\subsection{Input-Output pairing}
In order to determine the best control input to plant output pairing one can consult the Relative Gain Array ($RGA(G(s))$).
Since the RGA requires the system inverse, this is a good place to start.

\begin{equation}
G^{-1}(s) = 
\begin{pmatrix}
\frac{15}{50s + 1} && \frac{-11}{10s + 1} \\[6pt]
\frac{-25}{50s + 1} && \frac{70}{10s + 1}
\end{pmatrix}^{-1} = 
\begin{pmatrix}
\frac{14(50s + 1)}{155} && \frac{11(50s + 1)}{775} \\[6pt]
\frac{10s + 1}{31} && \frac{3(10s + 1)}{155}
\end{pmatrix}
\end{equation}

Where the operations has been carried out suing matlab.
Then the RGA can be calculated in matlab as well.

\begin{equation}
\begin{split}
RGA(G(s)) &= 
\begin{pmatrix}
\frac{15}{50s + 1} && \frac{-11}{10s + 1} \\[6pt]
\frac{-25}{50s + 1} && \frac{70}{10s + 1}
\end{pmatrix} .* 
\begin{pmatrix}
\frac{14(50s + 1)}{155} && \frac{11(50s + 1)}{775} \\[6pt]
\frac{10s + 1}{31} && \frac{3(10s + 1)}{155}
\end{pmatrix}^T = \\
RGA(G(s)) &= 
\begin{pmatrix}
\frac{67.74s + 1.355}{50s + 1} && \frac{-3.548s - 3548}{10s + 1} \\[6pt]
\frac{-17.74s - 0.3548}{50s + 1} && \frac{13.55s + 1.355}{10s + 1}
\end{pmatrix}
\end{split}
\end{equation} 

Then the pairing should be chosen such that the RGA element corresponding to the paring is not negative for $s=0$.
The RGA for $s=0$ is found by simple substitution.

\begin{equation}
RGA(G(0)) = 
\begin{pmatrix}
1.3548 && -0.3548 \\
-0.3548 && 1.3548
\end{pmatrix}
\end{equation}

This indicates that a pairing corresponding to the off-diagonal should be avoided.
In addition, the best pairing is achieved when the paring elements are close to $1$ on the imaginary axis.
This can be evaluated for the desired crossover frequency $\omega_c = 1/10$.

\begin{equation}
RGA(G(\omega_ci)) = 
\begin{pmatrix}
1.3548 && -0.3548 \\
-0.3548 && 1.3548
\end{pmatrix}
\end{equation}

This again indicates the better pairing is choosing the diagonal.
The chosen pairing is explicitly written out in the results, X)

\subsection{Decoupling}
In this part of the assignment the result from choosing the best input-output pairing shall be compared to using decoupling matrices $W_1$ and $W_2$.
In all cases $W_2 = I$, but $W_1$ shall be chosen $W_1 = G^{-1}(s)$ where there are two cases for $s$, $s_i = 0$ and $s_{ii}=\omega_ci$.
These cases represent steady-state decoupling (i) and dynamic decoupling (ii).
For the benchmark case no coupling is chosen $W_1 = I$, but merely the pairing.
The feedback controller is contrived of the decoupling matrices and a gain gain matrix $K = \begin{pmatrix} 1 && 0 \\ 0 && 1 \end{pmatrix}$.
The feedback control is constructed such as:

\begin{equation}
\begin{split}
u(t) &= F_y(r(t) - y(t)) \\
F_y &= W_1KW_2
\end{split}
\end{equation}

The explicit decoupling matrices used in the evaluation is calculated in Matlab and presented below:

\begin{equation}
\begin{split}
W^{(0)}_1 &= 
\begin{pmatrix}
1 && 0 \\ 0 && 1
\end{pmatrix} \\
W^{(i)}_1 &= G^{-1}(0) = 
\begin{pmatrix}
0.0903 && 0.0142 \\ 0.0323 && 0.0194
\end{pmatrix} \\
W^{(ii)}_1 &= Re(G^{-1}(\omega_ci)) = 
\begin{pmatrix}
2.3484 && 0.3690 \\ 0.0645 && 0.0387
\end{pmatrix}
\end{split}
\end{equation}

Were one can see that the decoupling matrix for $s=\omega_ci$ is complex valued so the real valued matrix have to be chosen for it to be realizable.
Now with the controllers defined for the different cases, they can be compared using their step responses, see figure X.


\subsection{Optimal control}

\section{Results}
This section states the results achieved.

\subsection{Input-Output pairing}
For the pairing requested, the RGA matrix was consulted and the best pairing is:

\begin{equation}
\begin{split}
u_1 \leftrightarrow y_1 \\
u_2 \leftrightarrow y_2
\end{split}
\end{equation}

\subsection{Decoupling}
TODO: Plots
\subsection{Optimal control}
TODO: Plots


\section{Discussion}



\section{Conclusion}

\clearpage
\bibliography{reference}

\clearpage
\appendix

\section{Example Section}
This is an example reference \citep{glad00}.

%\begin{figure}[h!]
%\center
%\includegraphics[scale=0.8]{../code/figures/exampleFigure.png}
%\caption{Example caption.}
%\label{fig:exampleLable}
%\end{figure}

\lstinputlisting{../code/main.m}

\end{document}