\documentclass[a4paper, titlepage]{article}

% For equations
\usepackage{amsmath}

% For including figures
\usepackage{graphicx}
\usepackage{float}

% Bibiliography setup
\usepackage[square]{natbib}
\bibliographystyle{agsm}
\usepackage[nottoc]{tocbibind}

% For typesetting matlab
\usepackage{listings}
\usepackage{color} %red, green, blue, yellow, cyan, magenta, black, white
\definecolor{mygreen}{RGB}{28,172,0} % color values Red, Green, Blue
\definecolor{mylilas}{RGB}{170,55,241}

\lstset{language=Matlab,%
    basicstyle=\small,
    breaklines=true,%
    frame = single,
    morekeywords={matlab2tikz},
    keywordstyle=\color{blue},%
    morekeywords=[2]{1}, keywordstyle=[2]{\color{black}},
    identifierstyle=\color{black},%
    stringstyle=\color{mylilas},
    commentstyle=\color{mygreen},%
    showstringspaces=false,
    numbers=left,%
    numberstyle={\tiny \color{black}},% size of the numbers
    numbersep=9pt, % this defines how far the numbers are from the text
    emph=[1]{for,end,break},emphstyle=[1]\color{red}, %some words to emphasise
    %emph=[2]{word1,word2}, emphstyle=[2]{style},    
}

\begin{document}

\begin{titlepage}
  \begin{center}
    \vspace*{1cm}
    \includegraphics[scale=1.0]{../figures/hig_logo_eng.png}\\
    \vspace{1.5cm}
    \large EEA004 - Multivariable and Nonlinear Control Systems\\
    \large Assignment 2\\
    \vspace{1.5cm}
    Group 4\\
    Dan Thilderkvist and Philip Gutierrez\\
    dan.thilderkvist@hotmail.com philipgutierrez67@gmail.com\\
    Files: main.m\\
    
    \vspace{1cm}
    \today
  \end{center}
\end{titlepage}


\section{Introduction}
This assignment is a continuation from a previous assignment where the dynamics of an air handling unit consisting of a heater and humidifier was analyzed.
In this report the air handling unit will be analyzed and controlled more in-depth.
That entails looking at more general methods for choosing input output control pairs, constructing decoupling matrices and using optimal state feedback to control the system.
The air handling system was modeled in the previous assignment but the plant transfer function is stated here for completeness:

\begin{equation}
G(s) = 
\begin{pmatrix}
\frac{15}{50s + 1} && \frac{-11}{10s + 1} \\[6pt]
\frac{-25}{50s + 1} && \frac{70}{10s + 1}
\end{pmatrix}
\end{equation}

\subsection{Theory}

A measurement of system input-output interaction is the Relative Gain Array (RGA).
The RGA for any matrix is defined as below \citep[~p.219]{glad00}:

\begin{equation}
RGA(A) = A.*(A^{-1})^T
\end{equation}

Here the multiplication is element-wise and the inverse can be exchanged for the pseudo-inverse if $A$ is not square.
For a system to be controlled by decoupled controllers the input-output pairs should be chosen such that the RGA matrix elements are non-negative for $s=0$ if A is a transfer function $G(s)$.

TODO: Part about decoupling matrices

A Linear Quadratic Regulator (LQR) is a state feedback controller that is optimal based on some defined cost.
Furthermore the Linear Quadratic Gaussian (LQG) is a state feedback controller that is optimal in the presence of system disturbances.
The criteria that is minimized by the LQG controller is:

\begin{equation}
V = \Vert z \Vert^2_{Q_1} + \Vert z \Vert^2_{Q_1}
\end{equation}

Where $z$ is the system true output, $u$ is the control action and $Q_1, Q_2$ are weight matrices that are used as design parameters.
The resulting controller is a gain matrix used for state feedback:

\begin{equation}
u(t) = -L\hat{x}
\end{equation}

Where $\hat{x}$ is the state estimate by the Kalman filter.
If the state can be measured the Kalman filter can be omitted, resulting in the LQR controller.
\citep[~p.242-247]{glad00}

\section{Method}
This section describes the methodology used to obtain the requested results in the assignment.
It will go through them step by step, starting by choosing an input-output pairing.

\subsection{Decentralization}
In order to determine the best control input to plant output pairing one can consult the Relative Gain Array ($RGA(G(s))$).
Since the RGA requires the system inverse, that is where the calculation start.

\begin{equation}
G^{-1}(s) = 
\begin{pmatrix}
\frac{15}{50s + 1} && \frac{-11}{10s + 1} \\[6pt]
\frac{-25}{50s + 1} && \frac{70}{10s + 1}
\end{pmatrix}^{-1} = 
\begin{pmatrix}
\frac{14(50s + 1)}{155} && \frac{11(50s + 1)}{775} \\[6pt]
\frac{10s + 1}{31} && \frac{3(10s + 1)}{155}
\end{pmatrix}
\label{equ:inverse}
\end{equation}

Where the operations has been carried out using Matlab.
Then the RGA can be calculated in Matlab as well.

\begin{equation}
\begin{split}
RGA(G(s)) &= 
\begin{pmatrix}
\frac{15}{50s + 1} && \frac{-11}{10s + 1} \\[6pt]
\frac{-25}{50s + 1} && \frac{70}{10s + 1}
\end{pmatrix} .* 
\begin{pmatrix}
\frac{14(50s + 1)}{155} && \frac{11(50s + 1)}{775} \\[6pt]
\frac{10s + 1}{31} && \frac{3(10s + 1)}{155}
\end{pmatrix}^T = \\
RGA(G(s)) &= 
\begin{pmatrix}
\frac{67.74s + 1.355}{50s + 1} && \frac{-3.548s - 3548}{10s + 1} \\[6pt]
\frac{-17.74s - 0.3548}{50s + 1} && \frac{13.55s + 1.355}{10s + 1}
\end{pmatrix}
\end{split}
\end{equation} 

Then the pairing should be chosen such that the RGA element corresponding to the paring is not negative for $s=0$.
The RGA for $s=0$ is found by simple substitution.

\begin{equation}
RGA(G(0)) = 
\begin{pmatrix}
1.3548 && -0.3548 \\
-0.3548 && 1.3548
\end{pmatrix}
\end{equation}

This indicates that a pairing corresponding to the off-diagonal should be avoided.
In addition, the best pairing is achieved when the paring elements are close to $1$ on the imaginary axis.
This can be evaluated for the desired crossover frequency $\omega_c = 1/10$.

\begin{equation}
RGA(G(\omega_ci)) = 
\begin{pmatrix}
1.3548 && -0.3548 \\
-0.3548 && 1.3548
\end{pmatrix}
\end{equation}

This again indicates the better pairing is choosing the diagonal.
The chosen pairing is explicitly written out in the results, X)

\subsection{Decoupling}
In this part of the assignment the result from choosing the best input-output pairing shall be compared to using decoupling matrices $W_1$ and $W_2$.
In all cases $W_2 = I$, but $W_1$ shall be chosen $W_1 = G^{-1}(s)$ where there are two cases for $s$, $s_i = 0$ and $s_{ii}=\omega_ci$.
These cases represent steady-state decoupling (i) and dynamic decoupling (ii).
For the benchmark case (0), no decoupling is chosen $W^{(0)}_1 = I$, but merely the pairing.
The feedback controller is contrived of the decoupling matrix $W_2$ and a gain gain matrix $K = \begin{pmatrix} 1 && 0 \\ 0 && 1 \end{pmatrix}$ and the decoupling matrix $W_1$ act in feedforward on the feedback error.
The control signal $u(t)$ is constructed such as:

\begin{equation}
\begin{split}
u(t) &= W_1(r(t) - F_yy(t)) \\
F_y &= KW_2
\end{split}
\label{equ:fbControl}
\end{equation}

To calculate the decoupling matrix $W_1$ the system inverse is required.
The inverse has already been calculated in (\ref{equ:inverse}), by simple substitution the decoupling matrices can be found.

\begin{equation}
\begin{split}
W^{(0)}_1 &= 
\begin{pmatrix}
1 && 0 \\ 0 && 1
\end{pmatrix} \\
W^{(i)}_1 &= G^{-1}(0) = 
\begin{pmatrix}
0.0903 && 0.0142 \\ 0.0323 && 0.0194
\end{pmatrix} \\
W^{(ii)}_1 &= Re(G^{-1}(\omega_ci)) = \\
&=
Re\begin{pmatrix}
0.0903 + 0.4516i && 0.0142 + 0.0710i \\
0.0323 + 0.0323i && 0.0194 + 0.0194i
\end{pmatrix} = \\
&= 
\begin{pmatrix}
0.0903 && 0.0142 \\ 0.0323 && 0.0194
\end{pmatrix}
\end{split}
\label{equ:decoupling}
\end{equation}

Since the inverse of $G$ is complex-valued for any non real $s$ the decoupling matrix is a complex-valued matrix.
In order for the decoupling to be realizable the real part is taken as expressed in (\ref{equ:decoupling}).
Now with the controllers defined for the three different cases, the closed loop system can be setup.

\begin{equation}
G_{cl} = \frac{GW_1}{I+GW_1F_y}
\label{equ:closedLoopDecoup}
\end{equation}

This results in three closed loop systems, but since $W^{i}_1 = W^{ii}_1$ two systems are identical.
The system has two reference inputs and 2 outputs.
In order to compare the decoupling methods the closed loop systems can be evaluated by a step response on input 1 ($r_1$) while input 2 ($r_2$) is zero.
The step responses and the control signal $u_1, u_2$ for the step response can be seen in figure \ref{fig:decouplingA}.

With a decoupled system, SISO control techniques can be employed.
Now the control structure (\ref{equ:fbControl}) can be augmented with two PI controllers $F_{PI}$ on the feedback error.

\begin{equation}
\begin{split}
u(t) &= F_{PI}W^{(ii)}_1(r(t) - F_yy(t)) \\
F_{PI} &= \frac{10s+1}{10s}
\begin{pmatrix}
1 && 0 \\ 0 && 1
\end{pmatrix} \\
F_y &= W_2K
\end{split}
\end{equation}

The new closed loop system is now:

\begin{equation}
G_{cl} = \frac{GW^{(ii)}_1F_{PI}}{I+GW^{(ii)}_1F_{PI}F_y}
\end{equation}

The resulting closed loop system can now be evaluated by the same step response, step on $r_1$ and zero on $r_2$.
The resulting system response and actuator signal can be seen in results, figure \ref{fig:decouplingB}.

\subsection{Optimal control}
Instead of performing explicit decoupling, a LQR controller can be used.
Since a LQR controller require the system states, it is convenient to express $G$ on state space form.
This was done in the previous assignment, the result is stated here again:

\begin{equation}
\begin{split}
\dot{x} &= 
\begin{pmatrix}
-0.02 & 0 \\ 0 & -0.1
\end{pmatrix}x
+
\begin{pmatrix}
1 & 0 \\ 0 & 4
\end{pmatrix}u \\
y &= 
\begin{pmatrix}
0.3 & -0.275 \\ -0.5 & 1.75
\end{pmatrix}x
\end{split}
\label{equ:oldSS}
\end{equation}

In this representation we are measuring temperature $y_1$ and humidity $y_2$ as per system definition.
However temperature and humidity is not the system states and we cannot use a state feedback controller without an observer that estimate the states.
A simpler approach in this case would be to modify the state space representation so that the states are temperature and humidity.
This can be done with by choosing new states $\tilde{x} = C^{-1}x$, if $C$ is invertible.
This is valid since a state space representation is arbitrary.
The new state space model is calculated by the Matlab command \verb|ss2ss(G_ss, G_ss.C)| and give the following model:

\begin{equation}
\begin{split}
\dot{\tilde{x}} &= 
\begin{pmatrix}
0.008387 && 0.01703 \\ -0.1806 && -0.1284
\end{pmatrix}\tilde{x}
+
\begin{pmatrix}
0.3 && -1.1 \\ -0.5 && 7
\end{pmatrix}u \\
y &= 
\begin{pmatrix}
1 & 0 \\ 0 & 1
\end{pmatrix}\tilde{x}
\end{split}
\label{equ:newSS}
\end{equation}

The new state $\tilde{x}$ now represent temperature and humidity and is the same as our measured output $y$.
The new model will be used for the LQR design and the tilde will be omitted form the new state hereafter.

To design an LQR controller is simple in Matlab through the command \verb|lqr(sys, Q1, Q2)|.
It requires the two weight matrices $Q_1$ and $Q_2$.
It is the relation between the weight matrices that set the cost balance between  reference tracking and actuation effort.
Three cases will be compared where $Q_1 = I$ and $Q_2 = \alpha I$.
The multiplier $\alpha$ will be chosen differently for the baseline and the two comparison cases.

\begin{equation}
\begin{split}
\alpha^{(0)} &= 1 \\
\alpha^{(i)} &= 10 \\
\alpha^{(ii)} &= 0.1
\end{split}
\end{equation}

Now the LQR controller can be designed using the state space model (\ref{equ:newSS}) and the weight matrices $Q_1$ and $Q_2$.
The state feedback controller returned from Matlab when minimizing the cost function will be denoted $L$.
Three such controllers $L^{(0)}, L^{(i)} \:\&\: L^{(ii)}$ are generated and the gains are as follow:

\begin{equation}
\begin{split}
L^{(0)} &= 
\begin{pmatrix}
0.9110 	&& 0.0685  \\
-0.1134 	&& 0.9775
\end{pmatrix} \\
L^{(i)} &= 
\begin{pmatrix}
3.0628 && 0.2451 \\
-0.2910 && 3.1324
\end{pmatrix} \\
L^{(ii)} &= 
\begin{pmatrix}
0.2393 && 0.0140 \\
-0.0557 && 0.2965
\end{pmatrix}
\end{split}
\label{equ:lqrCont}
\end{equation}

With a control topology with gain only on the feedback, the closed loops system is quite simple.

\begin{equation}
G_{cl} = \frac{G}{I + GL}
\end{equation}

The closed loop system is evaluated by the step response of a step on $r_1$ and zero on $r_2$ just as in previous evaluations.
The resulting system output and actuation can be seen in result, figure \ref{fig:lqgControl}.


\section{Results}
This section states the results achieved when comparing decentralization, decoupling and optimal control for a Multiple Input, Multiple Output system.

\subsection{Input-Output pairing}
For the pairing requested, the RGA matrix was consulted and the best pairing is:

\begin{equation}
\begin{split}
u_1 \leftrightarrow y_1 \\
u_2 \leftrightarrow y_2
\end{split}
\end{equation}

\subsection{Decoupling}

\begin{figure}[H]
\center
\includegraphics[scale=0.7]{../code/figures/decoupling_a.png}
\caption{Feedback without (blue) and with decoupling using unitary feedback controller. Steady-state decoupling (red) and approximate decoupling (yellow).}
\label{fig:decouplingA}
\end{figure}


\begin{figure}[H]
\center
\includegraphics[scale=0.7]{../code/figures/decoupling_b.png}
\caption{Feedback without (blue) and with decoupling using PI feedback controller. Steady-state decoupling (red) and approximate decoupling (yellow).}
\label{fig:decouplingB}
\end{figure}



\subsection{Optimal control}
TODO: Plots

\begin{figure}[H]
\center
\includegraphics[scale=0.7]{../code/figures/lqg.png}
\caption{Feedback using a LQG controller.}
\label{fig:lqgControl}
\end{figure}


\section{Discussion}



\section{Conclusion}

\clearpage
\bibliography{reference}

\clearpage
\appendix

\section{Example Section}
This is an example reference \citep{glad00}.

%\begin{figure}[h!]
%\center
%\includegraphics[scale=0.8]{../code/figures/exampleFigure.png}
%\caption{Example caption.}
%\label{fig:exampleLable}
%\end{figure}

\lstinputlisting{../code/main.m}

\end{document}