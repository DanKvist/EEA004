\documentclass[a4paper, titlepage]{article}

% For equations
\usepackage{amsmath}

% For including figures
\usepackage{graphicx}
\usepackage{float}

% Bibiliography setup
\usepackage[square]{natbib}
\bibliographystyle{agsm}
\usepackage[nottoc]{tocbibind}

% For typesetting matlab
\usepackage{listings}
\usepackage{color} %red, green, blue, yellow, cyan, magenta, black, white
\definecolor{mygreen}{RGB}{28,172,0} % color values Red, Green, Blue
\definecolor{mylilas}{RGB}{170,55,241}

\lstset{language=Matlab,%
    basicstyle=\small,
    breaklines=true,%
    frame = single,
    morekeywords={matlab2tikz},
    keywordstyle=\color{blue},%
    morekeywords=[2]{1}, keywordstyle=[2]{\color{black}},
    identifierstyle=\color{black},%
    stringstyle=\color{mylilas},
    commentstyle=\color{mygreen},%
    showstringspaces=false,
    numbers=left,%
    numberstyle={\tiny \color{black}},% size of the numbers
    numbersep=9pt, % this defines how far the numbers are from the text
    emph=[1]{for,end,break},emphstyle=[1]\color{red}, %some words to emphasise
    %emph=[2]{word1,word2}, emphstyle=[2]{style},    
}

\begin{document}

\begin{titlepage}
  \begin{center}
    \vspace*{1cm}
    \includegraphics[scale=1.0]{../figures/hig_logo_eng.png}\\
    \vspace{1.5cm}
    \large EEA004 - Multivariable and Nonlinear Control Systems\\
    \large Assignment 2\\
    \vspace{1.5cm}
    Group 4\\
    Dan Thilderkvist and Philip Gutierrez\\
    dan.thilderkvist@hotmail.com philipgutierrez67@gmail.com\\
    Files: main.m\\
    
    \vspace{1cm}
    \today
  \end{center}
\end{titlepage}

\tableofcontents
\clearpage

\section{Introduction}
This assignment is a continuation from a previous assignment where the dynamics of an air handling unit consisting of a heater and humidifier was analyzed.
In this report the air handling unit will be analyzed and controlled more in-depth.
That entails looking at more general methods for choosing input output control pairs, constructing decoupling matrices and using optimal state feedback to control the system.
The air handling system was modeled in the previous assignment but the plant transfer function is stated here for completeness:

\begin{equation}
G(s) = 
\begin{pmatrix}
\frac{15}{50s + 1} && \frac{-11}{10s + 1} \\[6pt]
\frac{-25}{50s + 1} && \frac{70}{10s + 1}
\end{pmatrix}
\label{equ:origianlG}
\end{equation}

\subsection{Theory}

A measurement of system input-output interaction is the Relative Gain Array (RGA).
The RGA for any matrix is defined as below \citep[~p.219]{glad00}:

\begin{equation}
RGA(A) = A.*(A^{-1})^T
\label{equ:RGA}
\end{equation}

Here the multiplication is element-wise and the inverse can be exchanged for the pseudo-inverse if $A$ is not square.
For a system to be controlled by decentralized controllers the input-output pairs should be chosen such that the RGA matrix elements are non-negative for $s=0$ if A is a transfer function $G(s)$.

If the RGA matrix doesn't indicate any good decentralized input-output pairs, decoupling can be used.
Decoupling is done by pre-/post-multiplying the plant by decoupling matrices $W_1 \& W_2$ to modify the plant dynamics.

\begin{equation}
\tilde{G}(s) = W_2(s)G(s)W_1(s)
\end{equation}

The decoupling matrices should be chosen such that $RGA(\tilde{G}(s))$ indicate desired properties for decentralized control.
This is of course achieved when $\tilde{G}(s)$ is diagonal and hence completely decoupled.
Such $W_i$ will be dynamic and requires very good knowledge about the model.
A more common, simpler choice is steady state decoupling where $\tilde{G}(0)$ is made diagonal by using constant, real-valued matrices $W_i$.
Another alternative is to make $\tilde{G}(\omega i)$ diagonal for some frequency $\omega$, normally the cross-over frequency. \citep[~p.226]{glad00}








A Linear Quadratic Regulator (LQR) is a state feedback controller that is optimal based on some defined cost.
Furthermore the Linear Quadratic Gaussian (LQG) is a state feedback controller that is optimal in the presence of system disturbances.
The criteria that is minimized by the LQG controller is:

\begin{equation}
V = \Vert z \Vert^2_{Q_1} + \Vert z \Vert^2_{Q_2}
\end{equation}

Where $z$ is the system true output, $u$ is the control action and $Q_1, Q_2$ are weight matrices that are used as design parameters.
The resulting controller is a gain matrix used for state feedback:

\begin{equation}
u(t) = -L\hat{x}
\end{equation}

Where $\hat{x}$ is the state estimate by the Kalman filter.
If the state can be measured the Kalman filter can be omitted, resulting in the LQR controller.
\citep[~p.242-247]{glad00}

\section{Method}
This section describes the methodology used to obtain the requested results in the assignment.
It will go through them step by step, starting by choosing an input-output pairing.

\subsection{Decentralization}
In order to determine the best control input to plant output pairing one can consult the Relative Gain Array (\ref{equ:RGA}).
Since the RGA requires the system inverse, that is where the calculation start.

\begin{equation}
G^{-1}(s) = 
\begin{pmatrix}
\frac{15}{50s + 1} && \frac{-11}{10s + 1} \\[6pt]
\frac{-25}{50s + 1} && \frac{70}{10s + 1}
\end{pmatrix}^{-1} = 
\begin{pmatrix}
\frac{14(50s + 1)}{155} && \frac{11(50s + 1)}{775} \\[6pt]
\frac{10s + 1}{31} && \frac{3(10s + 1)}{155}
\end{pmatrix}
\label{equ:inverse}
\end{equation}

Where the inverse operation has been carried out using Matlab \verb|G\eye(2)|.
The RGA can now be calculated in Matlab.

\begin{equation}
\begin{split}
RGA(G(s)) &= 
\begin{pmatrix}
\frac{15}{50s + 1} && \frac{-11}{10s + 1} \\[6pt]
\frac{-25}{50s + 1} && \frac{70}{10s + 1}
\end{pmatrix} .* 
\begin{pmatrix}
\frac{14(50s + 1)}{155} && \frac{11(50s + 1)}{775} \\[6pt]
\frac{10s + 1}{31} && \frac{3(10s + 1)}{155}
\end{pmatrix}^T = \\
RGA(G(s)) &= 
\begin{pmatrix}
\frac{67.74s + 1.355}{50s + 1} && \frac{-3.548s - 3548}{10s + 1} \\[6pt]
\frac{-17.74s - 0.3548}{50s + 1} && \frac{13.55s + 1.355}{10s + 1}
\end{pmatrix}
\end{split}
\end{equation} 

Then the pairing should be chosen such that the RGA element corresponding to the paring is not negative for $s=0$.
The RGA for $s=0$ is found by simple substitution.

\begin{equation}
RGA(G(0)) = 
\begin{pmatrix}
1.3548 && -0.3548 \\
-0.3548 && 1.3548
\end{pmatrix}
\label{equ:RBAs0}
\end{equation}

This indicates that a pairing corresponding to the off-diagonal should be avoided.
In addition, the best pairing is achieved when the paring elements are close to $1$ on the imaginary axis.
This can be evaluated for the desired crossover frequency $\omega_c = 1/10$.

\begin{equation}
RGA(G(\omega_ci)) = 
\begin{pmatrix}
1.3548 && -0.3548 \\
-0.3548 && 1.3548
\end{pmatrix}
\label{equ:RGAs10}
\end{equation}

Again, this indicates that the better pairing is done by choosing the diagonal.
The chosen pairing is explicitly written out in the results section (\ref{equ:RGApairing}).

\subsection{Decoupling}
In this part of the assignment the result from choosing the best input-output pairing shall be compared to using decoupling matrices $W_1$ and $W_2$.
In all cases $W_2 = I$, but $W_1$ shall be chosen $W_1 = G^{-1}(s)$ where there are two cases for $s$, $s^{(i)} = 0$ and $s^{(ii)}=\omega_ci$.
These cases represent steady-state decoupling (i) and dynamic decoupling (ii).
For the benchmark case (0), no decoupling is chosen $W^{(0)}_1 = I$, but merely the pairing.
The feedback controller is contrived of the decoupling matrix $W_2$ and a gain gain matrix $K = \begin{pmatrix} 1 && 0 \\ 0 && 1 \end{pmatrix}$ and the decoupling matrix $W_1$ act in feedforward on the feedback error.
The control signal $u(t)$ is constructed such as:

\begin{equation}
\begin{split}
u(t) &= W_1(r(t) - F_yy(t)) \\
F_y &= KW_2
\end{split}
\label{equ:fbControl}
\end{equation}

To calculate the decoupling matrix $W_1$ the system inverse is required.
The inverse has already been calculated in (\ref{equ:inverse}) and by simple substitution the decoupling matrices can be found.

\begin{equation}
\begin{split}
W^{(0)}_1 &= 
\begin{pmatrix}
1 && 0 \\ 0 && 1
\end{pmatrix} \\
W^{(i)}_1 &= G^{-1}(0) = 
\begin{pmatrix}
0.0903 && 0.0142 \\ 0.0323 && 0.0194
\end{pmatrix} \\
W^{(ii)}_1 &= Re(G^{-1}(\omega_ci)) = \\
&=
Re\begin{pmatrix}
0.0903 + 0.4516i && 0.0142 + 0.0710i \\
0.0323 + 0.0323i && 0.0194 + 0.0194i
\end{pmatrix} = \\
&= 
\begin{pmatrix}
0.0903 && 0.0142 \\ 0.0323 && 0.0194
\end{pmatrix}
\end{split}
\label{equ:decoupling}
\end{equation}

Since the inverse of $G$ is complex-valued for any non real $s$ the decoupling matrix is a complex-valued matrix.
In order for the decoupling to be realizable the real part is taken as expressed in (\ref{equ:decoupling}).
Now with the controllers defined for the three different cases, the closed loop system can be setup.

\begin{equation}
G_{cl} = \frac{GW_1}{I+GW_1F_y}
\label{equ:closedLoopDecoup}
\end{equation}

This results in three closed loop systems, but since $W^{i}_1 = W^{ii}_1$ two systems are identical.
The system has two reference inputs and 2 outputs.
In order to compare the decoupling methods the closed loop systems can be evaluated by a step response on input 1 ($r_1$) while input 2 ($r_2$) is zero.
The step responses and the control signal $u_1, u_2$ for the step response can be seen in figure \ref{fig:decouplingA}.

With a decoupled system, SISO control techniques can be employed.
Now the control structure (\ref{equ:fbControl}) can be augmented with two PI controllers $F_{PI}$ on the feedback error.

\begin{equation}
\begin{split}
u(t) &= W^{(ii)}_1F_{PI}(r(t) - F_yy(t)) \\
F_{PI} &= \frac{10s+1}{10s}
\begin{pmatrix}
1 && 0 \\ 0 && 1
\end{pmatrix} \\
F_y &= W_2K
\end{split}
\end{equation}

The new closed loop system is now:

\begin{equation}
G_{cl} = \frac{GW^{(ii)}_1F_{PI}}{I+GW^{(ii)}_1F_{PI}F_y}
\end{equation}

The resulting closed loop system can now be evaluated by the same step response, step on $r_1$ and zero on $r_2$.
The resulting system response and actuator signal can be seen in results, figure \ref{fig:decouplingB}.

\subsection{Optimal control}
Instead of performing explicit decoupling, an LQR controller can be used.
Since an LQR controller require the system states, it is convenient to express $G$ on state space form.
This was done in the previous assignment, the result is stated here:

\begin{equation}
\begin{split}
\dot{x} &= 
\begin{pmatrix}
-0.02 & 0 \\ 0 & -0.1
\end{pmatrix}x
+
\begin{pmatrix}
1 & 0 \\ 0 & 4
\end{pmatrix}u \\
y &= 
\begin{pmatrix}
0.3 & -0.275 \\ -0.5 & 1.75
\end{pmatrix}x
\end{split}
\label{equ:oldSS}
\end{equation}

In this representation we are measuring temperature $y_1$ and humidity $y_2$ as per system definition.
However temperature and humidity is not the system states and we cannot use a state feedback controller without an observer that estimate the states.
A simpler approach in this case would be to modify the state space representation so that the states are temperature and humidity.
This can be done with by choosing new states $\tilde{x} = C^{-1}x$, if $C$ is invertible.
This is valid since a state space representation is arbitrary.
The new state space model is calculated by the Matlab command \verb|ss2ss(G_ss, G_ss.C)| and give the following model:

\begin{equation}
\begin{split}
\dot{\tilde{x}} &= 
\begin{pmatrix}
0.008387 && 0.01703 \\ -0.1806 && -0.1284
\end{pmatrix}\tilde{x}
+
\begin{pmatrix}
0.3 && -1.1 \\ -0.5 && 7
\end{pmatrix}u \\
y &= 
\begin{pmatrix}
1 & 0 \\ 0 & 1
\end{pmatrix}\tilde{x}
\end{split}
\label{equ:newSS}
\end{equation}

The new state $\tilde{x}$ now represent temperature and humidity and is the same as our measured output $y$.
The new model will be used for the LQR design and the tilde will be omitted form the new state hereafter.

The design of an LQR controller is simple in Matlab through the command \verb|lqr(sys, Q1, Q2)|.
It requires the two weight matrices $Q_1$ and $Q_2$.
It is the relation between the weight matrices that set the cost balance between  reference tracking and actuation effort.
Three cases will be compared where $Q_1 = I$ and $Q_2 = \alpha I$.
The multiplier $\alpha$ will be chosen differently for the baseline and the two comparison cases.

\begin{equation}
\begin{split}
\alpha^{(0)} &= 1 \\
\alpha^{(i)} &= 10 \\
\alpha^{(ii)} &= 0.1
\end{split}
\end{equation}

Now the LQR controller can be designed using the state space model (\ref{equ:newSS}) and the weight matrices $Q_1$ and $Q_2$.
The state feedback controller returned from Matlab when minimizing the cost function will be denoted $L$.
Three such controllers $L^{(0)}, L^{(i)} \:\&\: L^{(ii)}$ are generated and the gains are as follow:

\begin{equation}
\begin{split}
L^{(0)} &= 
\begin{pmatrix}
0.9110 	&& 0.0685  \\
-0.1134 	&& 0.9775
\end{pmatrix} \\
L^{(i)} &= 
\begin{pmatrix}
3.0628 && 0.2451 \\
-0.2910 && 3.1324
\end{pmatrix} \\
L^{(ii)} &= 
\begin{pmatrix}
0.2393 && 0.0140 \\
-0.0557 && 0.2965
\end{pmatrix}
\end{split}
\label{equ:lqrCont}
\end{equation}

For a control topology with gain only on the feedback, the closed loop system is quite simple.

\begin{equation}
G_{cl} = \frac{G}{I + GL}
\end{equation}

The closed loop system is evaluated by the step response of a step on $r_1$ and zero on $r_2$ just as in previous evaluations.
The resulting system output and actuation can be seen in result, figure \ref{fig:lqgControl}.


\section{Results}
This section states the results achieved when comparing decentralization, decoupling and optimal control for a Multiple Input, Multiple Output system.

\subsection{Input-Output pairing}
The ideal pairing is provided by the RGA matrix (\ref{equ:RGAs10}) and is found to be:

\begin{equation}
\begin{split}
u_1 \leftrightarrow y_1 \\
u_2 \leftrightarrow y_2
\end{split}
\label{equ:RGApairing}
\end{equation}

\subsection{Decoupling}

The step response of decoupling is presented in figure \ref{fig:decouplingA} compared to the baseline without decoupling as a reference.
The actuator signal is also shown and the second input $u_2$ and second output $u_2$ of the MIMO system is the dashed lines.

The step response of decoupling, augmented by a PI controller is presented in figure \ref{fig:decouplingB}.
Again it is evaluated by a step response.

\begin{figure}[H]
\center
\includegraphics[scale=0.9]{../code/figures/decoupling_a.png}
\caption{Feedback with and without decoupling using a unitary feedback controller.}
\label{fig:decouplingA}
\end{figure}

\begin{figure}[H]
\center
\includegraphics[scale=0.9]{../code/figures/decoupling_b.png}
\caption{Feedback with decoupling using a PI controller.}
\label{fig:decouplingB}
\end{figure}



\subsection{Optimal control}

The results of using an LQR controller is presented in figure \ref{fig:lqgControl} with three different penalty matrices on the actuator effort. 

\begin{figure}[H]
\center
\includegraphics[scale=0.9]{../code/figures/lqg.png}
\caption{Feedback using an LQR controller with three different penalty matrices on the actuator effort.}
\label{fig:lqgControl}
\end{figure}


\section{Discussion}

Using the RGA it was discovered that the best pairing is the same as one would guess knowing there is a heater and humidifier.
That means using the heater for temperature control and humidifier for humidity control as would seem logical without the RGA evaluation.

Taking the control one step further with decoupling it was possible to compare with the results using pure decentralization.
It was noted that steady-state decoupling $s=0$ and dynamic decoupling $s=i/10$ produced the same decoupling matrices.
This was not initially expected.
However, closer review of the system transfer function inverse, $G^{-1}$, reminds us that it comprises of first-order polynomials only, and therefore, the results are the same after the imaginary part is dropped.
In contrast, however, a second or higher order transfer function in $G(s)$ would result in a change in the real part of G as well as the imaginary part and different results would be obtained.
The results of decoupling shown in figure \ref{fig:decouplingA} show only a single line for case (i) \& (ii) due to this.
The Baseline case with only pairing show slight couplings effect in figure \ref{fig:decouplingA} where $y_2$ goes negative as a reaction to the step input on temperature.
It also show a fairly quick response with a rise time around 10s.
With decoupling implemented the coupling effect into $y_2$ from the step on temperature are gone completely.
But the system is now considerably slower with a rise time around 50s, but also a much calmer actuator usage.
It is seen in the results that the DC gain of the closed loop system is not unity.
This can be mitigated by using a pre-filter $F_r$ to scale the closed loop gain, but was explicitly discouraged in the assignment.

Adding a PI controller for the decoupled system has the advantage of removing the steady state error without a pre-filter.
It also allow design of the closed loop dynamics and with the specified controller gains the rise time has been reduced to roughly 35s for the decoupled system.
However, the integral part of the PI controller has made the closed loop system second order and the system now exhibits overshoot and oscillation.
The system is with PI control still well decoupled as there is no noticeable effect on the humidity when commanding a temperature step.

Using the LQ control method for a state feedback design instead of decoupling matrices appears to be able to achieve decoupling to some extent.
Most visible in figure \ref{fig:lqgControl} is however the fact that when the penalty is low on control action the step response is faster.
The coupling effect when using the LQR is also notably higher when using a larger control action penalty as can be noted on the dashed lines for $y_2$ in figure \ref{fig:lqgControl}.
Noted in the the optimal control strategy is that for a higher control input penalty, the control signal is actually larger.
This is not intuitive at first glace but can be explained by the fact that the system is exposed to a step input.
The cost function optimized with the \verb|lqr| command in Matlab does by default optimize for a case where the expected output is 0 and the expected control input is also 0.
This is not the case for a step response and no method to modify this was found going though the Matlab documentation.
An interesting continuation would be to add the pre-filter that normalize the closed loop system to have unit DC gain.
This could better represent the control action required if the system would be implemented in practice.

\section{Conclusion}

It has been demonstrated that SISO feedback control can be utilized in a decentralized method for some systems by adequately choosing the best input-output pairing.
However, with good knowledge of the system model one can achieve even better SISO results by decoupling the system using decoupling matrices.
An alternative, when SISO control might not be sufficient, is to utilize Linear Quadratic control.
This has been demonstrated to also achieve decoupling to some extent based on the choice of penalty matrices.
LQ control is considered a Multiple Input, Multiple Output (MIMO) control method and is easily augmented to handle noise and disturbances, but can also be scaled to handle more complex systems.



\clearpage
\bibliography{reference}

\clearpage
\appendix

\section{Appendix}
Here is the Matlab code used to generate the results in the report and the figures.

\lstinputlisting{../code/main.m}

\end{document}