\documentclass[a4paper, titlepage]{article}

% For equations
\usepackage{amsmath}

% For including figures
\usepackage{graphicx}
\usepackage{float}

% Bibiliography setup
\usepackage[square]{natbib}
\bibliographystyle{agsm}
\usepackage[nottoc]{tocbibind}

% For typesetting matlab
\usepackage{listings}
\usepackage{color} %red, green, blue, yellow, cyan, magenta, black, white
\definecolor{mygreen}{RGB}{28,172,0} % color values Red, Green, Blue
\definecolor{mylilas}{RGB}{170,55,241}

\lstset{language=Matlab,%
    %basicstyle=\color{red},
    breaklines=true,%
    morekeywords={matlab2tikz},
    keywordstyle=\color{blue},%
    morekeywords=[2]{1}, keywordstyle=[2]{\color{black}},
    identifierstyle=\color{black},%
    stringstyle=\color{mylilas},
    commentstyle=\color{mygreen},%
    showstringspaces=false,%without this there will be a symbol in the places where there is a space
    numbers=left,%
    numberstyle={\tiny \color{black}},% size of the numbers
    numbersep=9pt, % this defines how far the numbers are from the text
    emph=[1]{for,end,break},emphstyle=[1]\color{red}, %some words to emphasise
    %emph=[2]{word1,word2}, emphstyle=[2]{style},    
}


%\title{Assignment 1\\
%System description and analysis\\
%\large EEA004}
%author{Dan Thilderkvist, Philip Gutierrez}

\begin{document}

\begin{titlepage}
\begin{center}
\vspace*{1cm}
\includegraphics[scale=1.0]{../figures/hig_logo_eng.png}\\
\vspace{1.5cm}
\large EEA004 - Multivariable and Nonlinear Control Systems\\
\large Assignment 1\\
\vspace{1.5cm}
Group 4\\
Dan Thilderkvist and Philip Gutierrez\\
dan.thilderkvist@hotmail.com philipgutierrez67@gmail.com\\
File: DanTPhilipGAssignment1.zip\\
\vspace{1cm}
\today
\end{center}
\end{titlepage}

%\maketitle


\section{Background/Introduction}
This assignment will study an air handling system with a heater and a humidifier as shown in figure \ref{fig:airSystem}.
The system allows control of two variables; temperature $y_1$ and humidity $y_2$.
This can be done through the flow of a heater $u_1$ and the flow of a humidifier $u_2$.

\begin{figure}[h!]
\center
\includegraphics[scale=1]{../figures/heaterHumidifier.png}
\caption{Air management system with heater $u_1$, humidifier $u_2$, thermometer $y_1$ and hygrometer $y_2$.}
\label{fig:airSystem}
\end{figure}

Although the system is not linear, it is assumed to be linear within a small region of operation.
With this in mind, both the heater and the humidifier can be modeled as first order systems (\ref{equ:firstOrder}).

\begin{equation}
G(s) = \frac{K}{\tau s + 1}
\label{equ:firstOrder}
\end{equation}

The time constant $\tau$ has been given for the heater as $\tau_1=50$ and for the humidifier as $\tau_2=10$ in the assignment.
The static gain $K$ will have to be calculated for each system.
For that purpose the assignment includes two enthalpy–entropy charts, one for each system.
There is also information provided how the system reacts to a step input of $u_1 = 1$ or $u_2 = 2$.
The heater, when fully opened ($u_1 = 1$) will increase the incoming air of $10^\circ C, 40\% \: RH$ by $15^\circ C$, and the humidifier, when fully opened ($u_2 = 1$), will increase the humidity of the incoming air of $20^\circ C, 10\% \: RH$ by $70\% \: RH$.

\subsection{Theory}
A linear system can be represented in state space form.
The general form of a state space formulation is as follows \citep[p.~31]{glad00}:

\begin{equation}
\begin{split}
\dot{x} &= Ax + Bu \\
y &= Cx + Du
\end{split}
\label{equ:stateSpace}
\end{equation}


The final value theorem is often used for system identification.
It is especially useful when a system is exited by a step input and the steady state output can be measured.
The final value theorem states\citep[p.~29]{glad00}:

\begin{equation}
\lim_{t \to \infty} f(t) = \lim_{s \to 0} sF(s)
\label{equ:finalTheorem}
\end{equation}

Observability of a system loosely indicates if the system states can be reconstructed from the output. If a system is observable the observer poles can be placed arbitrarily.
The observability matrix is defined as such \citep[p.~45]{glad00}:

\begin{equation}
\mathcal{O}(A,C) = 
\begin{pmatrix}
C \\ CA \\ \vdots \\ CA^{n-1}
\end{pmatrix}
\label{equ:observ}
\end{equation}

If the observability matrix $\mathcal{O}$ has full rank (independent rows/columns) the system is observable.

In a similar fashion, controllability loosely indicates if all states of a system can be controlled by the input. If a system is controllable the poles of a state feedback controller can be arbitrarily chosen.
The Controllability matrix is defined as such\citep[p.~31]{glad00}:

\begin{equation}
\mathcal{S}(A,B) = 
\begin{pmatrix}
B & AB & \cdots & A^{n-1}C
\end{pmatrix}
\label{equ:contr}
\end{equation}

If the controllability matrix $\mathcal{S}$ has full rank (independent rows/columns) the system is controllable.

When analyzing closed loop systems, not only is the closed loop transfer function $G_c$ of interest, but also a few other functions.
These other functions are related to output stability via noise, disturbance and modeling errors.
They are the sensitivity function $S$, the complementary sensitivity function $T$ and the input sensitivity function $S_u$.
They are defined as follows \citep[p.~149]{glad00}:

\begin{equation}
\begin{split}
G_c &= (I + GF_y)^{-1}GF_r \\
S &= (I + GF_y)^{-1} \\
T &= (I + GF_y)^{-1}GF_y \\
S_u &= (I + F_yG)^{-1}
\end{split}
\label{equ:transFunc}
\end{equation}

Here $F_y$ is the feedback controller and $F_r$ the feedforward controller of the standard controller form \citep[p.~147]{glad00}.

\section{Method}
This section will describe the work flow of identifying the air handling system, performing basic analysis of the identified system, and in the end, designing a controller for for it.

\subsection{System identification}
The air handling system described in the introduction is one of multiple input multiple output (MIMO).
Such a system will have a transfer function matrix $\boldsymbol{G}(s)$ from input to output. In this particular case it will be a $2x2$ matrix because of the two input $u_1, u_2$ and the two output $y_1, y_2$.
The system can be written as such:

\begin{equation}
\begin{split}
\boldsymbol{Y}(s) &= \boldsymbol{G}(s)\boldsymbol{U}(s) \leftrightarrow \\
\begin{pmatrix}
Y_1 \\ Y_2
\end{pmatrix}
&=
\begin{pmatrix}
G_{11}(s) & G_{12}(s) \\ G_{21}(s) & G_{22}(s)
\end{pmatrix}
\begin{pmatrix}
U_1(s) \\ U_2(s)
\end{pmatrix}
\end{split}
\end{equation}

Both the heater and humidifier are approximated by first order systems (\ref{equ:firstOrder}).
Hence, the transfer function elements can be written as:

\begin{equation}
G_{ij}(s) = \frac{K_{ij}}{\tau_{ij} s + 1}
\label{equ:firstorderTransFunc}
\end{equation}

Where $\tau_{11} = \tau_{21} = \tau_1 = 50$ and $\tau_{12} = \tau_{22} = \tau_2 = 10$ are given by the assignment and presented in the introduction.

In order to find $K_{ij}$ the final value theorem [\ref{equ:finalTheorem}] can be applied along with the steady state information from the introduction.

\begin{equation}
\begin{split}
\lim_{t \to \infty} y_{1}(t) &= 
\lim_{s \to 0} sY_{1}(s) = 
\lim_{s \to 0} s(G_{11}(s)U_{1}(s) + G_{12}(s)U_{2}(s)) \\
\lim_{t \to \infty} y_{2}(t) &= 
\lim_{s \to 0} sY_{2}(s) = 
\lim_{s \to 0} s(G_{21}(s)U_{1}(s) + G_{22}(s)U_{2}(s))
\end{split}
\label{equ:finalUsage}
\end{equation}

If the input $u(t)$ is a unit step function $l(t)$, the transfer function $U(s)$ is $\frac{1}{s}$ and if only one of $u_1(t)/u_2(t)$ is a step and the other is zero, then one of the terms on the right hand side of equation [\ref{equ:finalUsage}] will disappear.
This will render 2 scenarios for the 2 equations:

\begin{equation}
\begin{split}
\lim_{t \to \infty} y_{1}(t) &= 
\lim_{s \to 0} G_{11}(s) = G_{11}(0) = K_{11}, 
\begin{cases}
 u_1(t) = l(t) \\ u_2(t) = 0 
\end{cases} \\
\lim_{t \to \infty} y_{2}(t) &= 
\lim_{s \to 0} G_{21}(s) = G_{21}(0) = K_{21}, 
\begin{cases}
 u_1(t) = l(t) \\ u_2(t) = 0 
\end{cases} \\
\lim_{t \to \infty} y_{1}(t) &= 
\lim_{s \to 0} G_{12}(s) = G_{12}(0) = K_{12}, 
\begin{cases}
 u_1(t) = 0 \\ u_2(t) = l(t) 
\end{cases} \\
\lim_{t \to \infty} y_{2}(t) &= 
\lim_{s \to 0} G_{22}(s) = G_{22}(0) = K_{22}, 
\begin{cases}
 u_1(t) = 0 \\ u_2(t) = l(t) 
\end{cases} \\
\end{split}
\end{equation}

A step input on only one input correspond to the excitation described in the assignment and was presented in the introduction of this report.
The output of such a step input was also given and corresponds to the output of 2 of the 4 equations.
The other 2 outputs can be found using an entropy-enthalpy chart and can be seen in figure \ref{fig:entropy}.
Here, the effect of a step input on the heater (red) and the water mist humidifier (green) can be seen.
The water humidifier was chosen in favor of the steam humidifier as the assignment didn't specify which to use.

\begin{figure}
\center
\includegraphics[scale=0.7]{../figures/enthalpyEntropy.png}
\caption{Effect on temperature and humidity by a heater step input (red) and of a humidifier step input (green).}
\label{fig:entropy}
\end{figure}

Reading off the temperature and humidity values from the chart give the final static values of the left hand side of equation [\ref{equ:finalUsage}].

\begin{equation}
\begin{split}
25 - 10 = 15 &= 
\lim_{t \to \infty} y_{1}(t) = 
K_{11} \\
15 - 40 = -25 &= 
\lim_{t \to \infty} y_{2}(t) = 
K_{21} \\
9 - 20 = -11 &= 
\lim_{t \to \infty} y_{1}(t) = 
K_{12} \\
80 - 10 = 70 &= 
\lim_{t \to \infty} y_{2}(t) = 
K_{22} \\
\end{split}
\end{equation}

With the values found for $K_{ij}$ and $\tau_{ij}$, the full MIMO system transfer function $\boldsymbol{G}(s)$ can be assembled and is presented in results as equation [\ref{equ:airPlantResult}.

\subsection{System state space formulation}
\label{sec:stateSpace}
A linear MIMO system such as the one developed in the previous example can also be expressed in state space form [\ref{equ:stateSpace}].
A simple way to do this is to take aid of a computer.
Matlab offers this functionality through the function \verb|ss(tf)|.
The resulting state space formulation of our system is presented under results as equation [\ref{equ:ssSolution}]

\subsection{Plant model singular values}
\label{sec:plantSingularValues}
The singular values $\sigma_i$ of a MIMO system in state space form [\ref{equ:stateSpace}] are the square roots of the eigenvalues $\sqrt{\lambda_i}$ of the matrix $A^*A$.
The singular values is a measurement of the gain of the system and the input amplification lies between the largest and smallest singular value.
Hence plotting the minimum and maximum singular values for all frequencies correspond to the MIMO equivalent of the bode gain plot.
Since this is tedious to do by hand, Matlab offers the function \verb|sigma(G)| for this purpose.
The singular value plot for the air handling system can be found under results as figure \ref{fig:systemsingular}.

\subsection{System observability and controllability}
For the air handling system the observability and controllability matrices [\ref{equ:observ}],[\ref{equ:contr}] can be calculated using the Matlab commands \verb|obsv(A,C)| and \verb|ctrb(A,B)|.
The matrices for the air handling system are:

\begin{equation}
\mathcal{O} = 
\begin{pmatrix}
0.3 & -0.275		\\
-0.5 & 1.75		\\
-0.006 & 0.0275	\\
0.01 & -0.175
\end{pmatrix}
,\;
\mathcal{C} = 
\begin{pmatrix}
1 & 0 & -0.02 & 0 \\
0 & 4 & 0 & -0.4
\end{pmatrix}
\end{equation}

Using the Matlab command \verb|rank(A)| provides the rank of these matrices.
The rank can then be compared to the smaller of column count or row count to see if the matrices are full rank.
If they are both full rank the plant is both observable and controllable.

\subsection{Controller design}
A controller $u(t)$ will be designed for the system using feedback.
The general form of a controller is \citep[p.~147]{glad00}:

\begin{equation}
u(t) = -K_y(p)y(t) + K_r(p)r(t)
\end{equation}

The assignment asks for a standard proportional controller on each output.
For a standard proportional controller $K_r = K_y$, and the proportional gains $k_{p,i}$ will be the diagonal of the controller matrix $K_y$.
With 2 system outputs, 2 control signals and 2 reference values, the size of $K_y$ will be a $2 \times 2$ matrix.
For the air handling system with proportional controller feedback, the controller will be:

\begin{equation}
\begin{split}
u(t) = -K_y(p)y(t) + K_y(p)r(t) &= 
K_y(p)(-y(t) + r(t)) = 
K_y(p)e(t) \\
K_y(p) &= 
\begin{pmatrix}
k_{p,1} 	& 	0 		\\
0 		&	k_{p,2}	
\end{pmatrix}
\end{split}
\end{equation}

The gain values $k_{p,i}$ are selected in 3 different interesting combinations, shown in table \ref{tab:gains}.

\begin{table}[h!]
\begin{center}
\begin{tabular}{||c | c c||}
 \hline
 Test & $k_{p,1}$ & $k_{p,2}$ \\ [0.5ex] 
 \hline\hline
 1 &  1 & 1 \\ 
 \hline
 2 &  1 & 0.1 \\
 \hline
 3 &  1 & 10 \\
 \hline
\end{tabular}
\end{center}
\caption{Controller gains used during evaluation.}
\label{tab:gains}
\end{table}

With the controller defined, the stability functions [\ref{equ:transFunc}] can be calculated using Matlab standard matrix operations.
The singular values of the stability functions are then plotted, again using \verb|sigma(F)|, the singular value plots are found in results as figures \ref{fig:plotCL},\ref{fig:plotS},\ref{fig:plotT} and \ref{fig:plotSu}.

\section{Results}

\subsection{Question 1}
From (8), (10), and (11) above, the air handling plant system transfer function is found to be:

\begin{equation}
G(s)
=
\begin{pmatrix}
\frac{15}{50s + 1} 	& 	\frac{-11}{10s + 1} \\[6pt]
\frac{-25}{50s + 1} 	& 	\frac{70}{10s + 1}
\end{pmatrix}
\label{equ:airPlantResult}
\end{equation}

\subsection{Question 2}
Following the method described in section \ref{sec:stateSpace} the state space representation of the air handling plant is found to be:

\begin{equation}
\begin{split}
\dot{x} &= 
\begin{pmatrix}
-0.02 & 0 \\ 0 & -0.1
\end{pmatrix}x
+
\begin{pmatrix}
1 & 0 \\ 0 & 4
\end{pmatrix}u \\
y &= 
\begin{pmatrix}
0.3 & -0.275 \\ -0.5 & 1.75
\end{pmatrix}x
\end{split}
\label{equ:ssSolution}
\end{equation}

\subsection{Question 3}
From section \ref{sec:plantSingularValues} the singular values of the air handling plant are shown in figure \ref{fig:systemsingular}:


\begin{figure}[h!]
\center
\includegraphics[scale=0.7]{../code/figures/systemSingular.png}
\caption{The singular values of the air handling plant.}
\label{fig:systemsingular}
\end{figure}


\subsection{Question 4}

The system is controllable as the controllability matrix has the same rank as number of rows, ie. full rank:

\begin{equation}
rank
\begin{pmatrix}
1.0 & 0 & -0.02 & 0 \\ 0 & 4.0 & 0 & -0.4
\end{pmatrix}
= 2
\label{equ:actual_contr}
\end{equation}


The system is observable as the observability matrix has the same rank as number of columns, ie. full rank:

\begin{equation}
rank
\begin{pmatrix}
0.3 & -0.275 \\ -0.5 & 1.75 \\ -0.006 & 0.0275 \\ 0.01 & -0.175
\end{pmatrix}
= 2
\label{equ:actual_observ}
\end{equation}


\subsection{Question 5}
The singular values for the stability functions can be seen in figure \ref{fig:plotCL},\ref{fig:plotS},\ref{fig:plotT} and \ref{fig:plotSu}.
In addition, the poles of the stability functions [\ref{equ:transFunc}] are presented in table \ref{tab:poles}.

\begin{table}[h!]
\begin{center}
\begin{tabular}{||c | c ||}
 \hline
 Controller & System poles ($G_c, S, T, S)u$) \\ [0.5ex] 
 \hline\hline
 $k_{p,1} = 1, k_{p,2} = 1$ &  $s_1 \approx -7.2, \; s_2 \approx -0.24$ \\ 
 \hline
 $k_{p,1} = 1, k_{p,2} = 0.1$ &  $s_1 \approx -0.90, \; s_2 \approx -0.22$ \\
 \hline
 $k_{p,1} = 1, k_{p,2} = 10$ &  $s_1 \approx -70, \; s_2 \approx -0.24$ \\
 \hline
\end{tabular}
\end{center}
\caption{System poles of the stability functions for the different controller gains.}
\label{tab:poles}
\end{table}


\begin{figure}[H]
\center
\includegraphics[scale=0.7]{../code/figures/plotCL.png}
\caption{Singular values of the closed loop system with three different proportional controllers.}
\label{fig:plotCL}
\end{figure}

\begin{figure}[H]
\center
\includegraphics[scale=0.7]{../code/figures/plotS.png}
\caption{Singular values of the sensitivity function with three different proportional controllers.}
\label{fig:plotS}
\end{figure}

\begin{figure}[H]
\center
\includegraphics[scale=0.7]{../code/figures/plotT.png}
\caption{Singular values of the complementary sensitivity function with three different proportional controllers.}
\label{fig:plotT}
\end{figure}

\begin{figure}[H]
\center
\includegraphics[scale=0.7]{../code/figures/plotSu.png}
\caption{Singular values of the input sensitivity function with three different proportional controllers.}
\label{fig:plotSu}
\end{figure}


\section{Conclusion/Discussion}

\subsection{System Identification}

%rease incoming air of $20^\circ C, 10\% \: RH$ by $70\% \: RH$.


The system is described as an air handling unit that consists of heater and a humidifier.
%The incoming air temperature is said to be $10^\circ C$ with a $40\% R.H.$ 
%While not stated in the assignment, it is interesting to note that the ideal relative humidity for health and comfort is between 20-50 percent R.H (mayo clinic) with a recommended minimum temperature of 18 degrees C (WHO2018).
%Therefore, the incoming air will likely need to be heated.\\    

It is provided that the heater and humidifier could both be described as first-order systems (\ref{equ:firstorderTransFunc}).
Each of the system outputs (temperature and humidity) can be controlled directly.  
However, by reviewing the entropy-enthalpy chart provided (\ref{fig:entropy}), it is noticed that each will have a negative impact on the other.  
That is, an increase in temperature will result in an undesirable decrease in humidity.  
Likewise, an increase in humidity levels, by spraying a cold water mist, will result in an undesirable decrease in temperature.\\

As a consequence, the act of heating and humidifying cannot be treated as two separate, and independent, actions, but instead, the two actions are coupled, and the resultant transfer function reflects this (\ref{equ:airPlantResult}).\\

The steady state gain of the system is obtained by employing the Final Value Theorem (\ref{equ:finalTheorem}) and obtaining the change in temperature and humidity from the entropy-enthalpy chart(fig. \ref{fig:entropy}).  A change in input $u_1$ from 0 to 1 corresponds to a positive change in temperature of $15^\circ C$ ($K_{11}$).  However, the temperature is also affected by a change in input $u_2$ by reducing the temperature by $11^\circ C$ ($K_{12}$).  The humidity is affected in a similar manner where a change in $u_1$ will counteract the desired set point specified by $u_2$.       

%It is interesting to note that an increase of $15^\circ C$ ($K_{11}$) will also result in a decrease of $25\% R.H.$ ($K_{21}$), while an increase in R.H. of $70\%$ ($K_{22}$) will result in a decrease in temperature of $11^\circ C$ ($K_{12}$).  Thus, the two actions are indeed counteracting each other as expected.\\

All first-order systems with a positive time constant will have poles strictly in the left-hand plane, hence, the system is stable even without a controller.
 
A further observation is the effect of choosing a cold-water mist versus a steam mist for humidifying.  The scenario explained above is based on a cold-water mist.  However, a steam based humidifier would result in a lateral shift in the entropy-enthalpy chart maintaining a constant temperature during the humidifying processes.  This would effectively set $K_{12}$ to zero.  While this would simplify the control algorithm, it would add the complexity and cost of a steam-based humidifier.\\   

\subsection{System State Space Formulation}
   
The state space representation of a system, in contrast to the transfer function representation, provides a transition from the frequency domain to the time domain \citep [p. 3]{williams07}.  This form of representation has the advantage that state space models can usually be obtained directly from the underlying physical models of the system \citep[p.~32]{glad00}.  An additional motivation is that in MIMO systems, one can convert a system of k n-th order differential equations into a system of kn coupled first-order differential equations \citep[p.~5]{williams07}.\\

The concept of state coordinate transformation allows a liner transformation of the state vector that yields a different state equation which also meets the systems overall input-output relationship \citep[p.~72]{williams07}.  Consequently, a state-space representation is not unique, and the state-space representation in (\ref{equ:ssSolution}) in simply one representation.\\

While the state of a system is indeed a physical concept, and unique at any given time, the means of representing the state is not unique, and any set of state variables may be used to represent the system \cite [p. 43]{melsa69}.  Therefore, the state vector may not necessarily represent physical quantities that can be measured directly. As an example, the state-space model could have been derived directly from the physical plant description(\ref{fig:airSystem}), in which case, the state vector could have been such that the state variables represent the physical quantities of temperature and humidity. 
TODO: I cannot find a reference to that C must be invertible.\\

\subsection{Plant Model Singular Values}

The singular values plot is shown in Figure \ref{fig:systemsingular}.  All inputs are damped at frequencies above approximately 0.1 rad/s, and the system gain decreases to zero as expected for a first order system.

By varying the two time constants it is observed that the attenuation of the upper curve seems to be dictated by the smaller time constant $\tau_{2}=10$ of the humidifier, while the lower curve is dictated by the larger time constant $\tau_{1}=50$ of the heater. 

\subsection{System observability and controllability}

The observability matrix has full rank (\ref{equ:actual_observ}) which means that all states may influence the outputs.

The controllability matrix has full rank (\ref{equ:actual_contr}) which means that all states can be stimulated by at least one input.

This state representation (\ref{equ:ssSolution}) is also a minimal realization as it is both observable and controllable, hence, the state vector is of the smallest possible dimension \cite [p. 46]{glad00}.

\subsection{Transfer Function of the Closed Loop System}
From table \ref{tab:poles} it can be seen that the closed loop system is stable for all controllers. 
Since all the stability functions [\ref{equ:transFunc}] has (almost) the same characteristic polynomial, they are all stable.
With a stable system one can draw conclusions from the singular value plot.

The closed loop singular value plot \ref{fig:plotCL} show the gain decrease to zero at high frequency as is typical for a first order system.
It also show that the low frequency gain (static gain) is not equal to 1 ($0dB$).
It is below $0dB$ and the upper and lower singular value bound has a range that is nonzero.
This will result in a steady state error of some magnitude for most input combinations (Controller 2 (red) actually encloses $0dB$ and could have unity static gain for some input).
This is a very common scenario for a pure proportional controller such as the one used in this assignment.

The sensitivity singular value plot \ref{fig:plotS} show how system disturbances affect the system output.
It can be seen that higher frequency disturbance propagate through the system while low frequency noise is attenuated.
Controller 2 (red line) even has a region where the disturbance gets amplified.
Since plant disturbances are usually of low frequency nature this is a good result and the controller farthest to the right is most advantageous from a disturbance rejection point of view.

The complementary sensitivity plot \ref{fig:plotT} is exactly the same as the closed loop.
This is due to the controller choice $K_y = K_r$.
This plot show how measurement noise is propagated though the system.
Measurement noise is usually of high frequency character.
Due to this, the low pass characteristic of the plot is desired and the controller furthest to the left is best from a noise rejection point of view.
What has been mentioned here highlights the conflict between disturbance rejection and noise rejection as the former benefit from a high rise frequency and the latter from a low cutoff frequency.

The last plot \ref{fig:plotSu} is the input sensitivity function.
It indicate how noise on the input affect the system output.
The singular value plot here indicate that controller 2 (red) could amplify such noise in a region $0.2 - 10 rad/s$

What choice of controller is best in practice come down to how the different sources of noise and disturbance behave in the real world system.
If the sensors are cheap and very noisy one might prefer  controller 2 (red) with the better noise rejection.
If instead the room being controlled suffer frequent door and window openings, controller 3 with the better disturbance rejection would be preferable.

\subsection{Dummy}
%This is a cnotinuation for chapter 4.2
In fact, if one so desires the concluded state space representation could be translated to one where the states have the same physical meaning as the measured output temperature and humidity. 
Since the state space representation is not unique, a state transformation can be done:

\begin{equation}
\tilde{x} = Cx
\end{equation}

This, assuming C is invertible $x = C^{-1}\tilde{x}$ (plausible for a square matrix) would generate the new and equivalent state space representation:

\begin{equation}
\begin{split}
\dot{\tilde{x}} &= C\dot{x} = CAC^{-1}\tilde{x} + CBu \\
y &= Cx = CC^{-1}\tilde{x} = \tilde{x}
\end{split}
\end{equation}

\clearpage

\bibliography{reference}
\clearpage

\appendix

\section{Code}
All code used for the assignment is included in this section.
It has been sorted per question the code relates to (1-5).

\subsection{Question 1}
\lstinputlisting{../code/question1.m}
\clearpage

\subsection{Question 2}
\lstinputlisting{../code/question2.m}
\clearpage

\subsection{Question 3}
\lstinputlisting{../code/question3.m}
\clearpage

\subsection{Question 4}
\lstinputlisting{../code/question4.m}
\clearpage

\subsection{Question 5}
\lstinputlisting{../code/question5.m}
\clearpage

\end{document}