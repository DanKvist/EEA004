\documentclass[a4paper, titlepage]{article}

% For equations
\usepackage{amsmath}

% For including figures
\usepackage{graphicx}
\usepackage{float}

% Bibiliography setup
\usepackage[square]{natbib}
\bibliographystyle{agsm}
\usepackage[nottoc]{tocbibind}

% For typesetting matlab
\usepackage{listings}
\usepackage{color} %red, green, blue, yellow, cyan, magenta, black, white
\definecolor{mygreen}{RGB}{28,172,0} % color values Red, Green, Blue
\definecolor{mylilas}{RGB}{170,55,241}

\lstset{language=Matlab,%
    basicstyle=\small,
    breaklines=true,%
    frame = single,
    morekeywords={matlab2tikz},
    keywordstyle=\color{blue},%
    morekeywords=[2]{1}, keywordstyle=[2]{\color{black}},
    identifierstyle=\color{black},%
    stringstyle=\color{mylilas},
    commentstyle=\color{mygreen},%
    showstringspaces=false,
    numbers=left,%
    numberstyle={\tiny \color{black}},% size of the numbers
    numbersep=9pt, % this defines how far the numbers are from the text
    emph=[1]{for,end,break},emphstyle=[1]\color{red}, %some words to emphasise
    %emph=[2]{word1,word2}, emphstyle=[2]{style},    
}


%\title{Assignment 3\\
%System description and analysis\\
%\large EEA004}
%\author{Dan Thilderkvist, Philip Gutierrez}

\begin{document}

%\maketitle

\begin{titlepage}
  \begin{center}
    \vspace*{1cm}
    \includegraphics[scale=1.0]{../figures/hig_logo_eng.png}\\
    \vspace{1.5cm}
    \large EEA004 - Multivariable and Nonlinear Control Systems\\
    \large Assignment 3\\
    \vspace{1.5cm}
    Group 4\\
    Dan Thilderkvist and Philip Gutierrez\\
    dan.thilderkvist@hotmail.com philipgutierrez67@gmail.com\\
    Files: main.m\\
    
    \vspace{1cm}
    \today
  \end{center}
\end{titlepage}

\tableofcontents
\clearpage



\section{Introduction}

This assignment is a continuation of the previous two assignments where an air handler is analyzed.
Here we will look closer at two of the non-linear components of the system, namely the heater and the valve that controls it.
Both the heater and the valve have non-linear, but complementary, characteristics which compensate for each other.  However, the valve also exhibits valve authority effect which compromises the compensation between heater and valve.  These non-linearities will be analyzed closer. 
\citep[p.123]{glad00}

\subsection{Theory}

\subsection{Describing Function}

The describing function is defined to be \citep[p. 358]{glad00}:

\begin{equation}
	Y_{f}(C) = \frac{b(C)+ia(C)}{(C}
	\label{equ:descrbingFunction}
\end{equation}

where $a(C)$ and $b(C)$ are the Fourier coefficients provided by:

\begin{equation}
\begin{split}
	a(C) = A(C)sin( phi(C) ) \\
	b(C) = A(C)con( phi(C) )
\end{split}
\label{equ:fourier}
\end{equation}

Self-sustained oscillation may occur if the following relationship is met \citep[p. 359]{glad00}:

\begin{equation}
	Y_{f}(C)G(iw) = -1
\label{equ:selfOsc}
\end{equation}
 
\section{Method}
\subsection{System Transfer Function}
The system to be analyzed is provided and is specified to be:

\begin{equation}
	G(s) = \frac{1}{(s^2+s+1)(s+3)}
	\label{equ:system}
\end{equation}

The system will be controlled by a valve which is considered to be static.  This valve has a gain $K_vs$, known as the "valve coefficient", and is available only with specific values.
With the addition of the gain $K_vs$ the new dynamic system becomes:

\begin{equation}
	G_{kv}(s) = \frac{K_{vs}}{(s^2+s+1)(s+3)} for K_{vs}=1, 1.6, 2.5, 4, 6.3, 10, and 16.
	\label{equ:systemTF}
\end{equation}


\subsection{Circle Criteria}

The method of Circle Criteria will be used to identify the valve with maximum power.  The exact nonlinearity is not known, but it is known to be limited to be inside the region specified by the blue arches as seen in figure \ref{fig:valvepower}.

\begin{figure}[h!]
\center
\includegraphics[scale=0.25]{../figures/valveOutputPower.png}
\caption{Valve Output Power}
\label{fig:valvepower}
\end{figure}

To apply the circle criteria, the region of known nonlinearity is bounded by two lines with slopes $k_{1}$ and $k_{2}$.  By inspection, these slopes are chosen to bound a region that includes the known region of nonlinearity of the valves.  From figure \ref{fig:valvepower} we find that:

\begin{equation}
k_{1} = \frac{1-0}{0.6-0} = \frac{1}{0.6}
\label{equ:k1_value}
\end{equation}

\begin{equation}
k_{2} = \frac{0.6-0}{1-0} = 0.6
\label{equ:k2_value}
\end{equation}

The values of $k_{1}$ and $k_{2}$ are used to form an exclusion region in the Nyquist diagram bounded by a circle which intersects the real axis at 1/$k_{1}$ and 1/$k_{2}$.  See Nyquist diagrams in the results section.

\subsection{Describing Function}

An ideal relay can be modeled as \citep[p. 358]{glad00}:

\begin{equation}
\begin{split}
f(e) = 1 if e > 0 \\
f(e) = -1 if e <0
\end{split}
\label{equ:relay}
\end{equation}

By using \ref{equ:descrbingFunction} and \ref{equ:fourier} the resulting describing function of the relay can be found to be:

\begin{equation}
Y_{f}(C) = \frac{4}{piC}
\label{equ:relayDescribingFuntion}
\end{equation}

For oscillation to occur, the relationship \ref{equ:selfOsc} must be satisfied.  Since the describing function of the relay \ref{equ:relayDescribingFuntion} is real and positive, we observe that $G(iw)$ in \ref{equ:selfOsc} must be real and negative.

From the system transfer function \ref{equ:systemTF} above we see that :

\begin{equation}
\begin{split}
G(iw) = \frac{K_{vs}}{((iw)^2+(iw)+1)((iw)+3)}\\
= \frac{K[(3-4w^2)-i(4w-w^3)]}{[(3-4w^2)+i(4w-w^3)][(3-4w^2)-i(4w-w^3)]}
\end{split}
\label{equ:systemTFiw}
\end{equation}

This equation is real when:

\begin{equation}
4-w^2 = 0 => w=2
\label{equ:systemTFreal}
\end{equation}

At this frequency we find that equation \ref{equ:systemTFiw} yields:

\begin{equation}
G(i2) = \frac{-K}{13}
\label{equ:systemTFw2}
\end{equation}

Finally, with the describing function of the relay specified in  (\ref{equ:relayDescribingFuntion}) and $G(iw)$ evaluated at the oscillating frequency $w=2$ provided in (\ref{equ:systemTFreal}) we can use the relationship from (\ref{equ:selfOsc}) to obtain:

\begin{equation}
Y_{f}(C)G(iw) = \frac{4}{piC}\frac{-K}{13} = -1
\label{equ:systemTFw2}
\end{equation}

The assignment asks for a magnitude of oscillation below $C < 0.5$.
Therefore, from (\ref{equ:systemTFw2}) above we find the maximum value of $K_{vs}$ to be:

\begin{equation}
K_{MAX} < \frac{13Cpi}{4} = 5.10 for C=0.5
\label{equ:maximum_k}
\end{equation}

\section{Results}

\subsection{Circle Criteria Results}
The following Nyquist diagrams include a circle of exclusion which is bounded by $1/k_{1}$ and $1/k_{2}$.  The diagrams also include an exclusion region bounded by $1/k_{2}$ (shown as a vertical line in the graphs) which is used when analyzing the case with saturation in the valve.

\begin{figure}[h!]
\center
\includegraphics[scale=0.25]{../code/figures/nyquist1.png}
\caption{Nyquist diagram for $K_{vs}=1.0$}
\label{fig:nyquist1}
\end{figure}

\begin{figure}[h!]
\center
\includegraphics[scale=0.25]{../code/figures/nyquist2.png}
\caption{Nyquist diagram for $K_{vs}=1.6$}
\label{fig:nyquist2}
\end{figure}

\begin{figure}[h!]
\center
\includegraphics[scale=0.25]{../code/figures/nyquist3.png}
\caption{Nyquist diagram for $K_{vs}=2.5$}
\label{fig:nyquist3}
\end{figure}

\begin{figure}[h!]
\center
\includegraphics[scale=0.25]{../code/figures/nyquist4.png}
\caption{Nyquist diagram for $K_{vs}=4.0$}
\label{fig:nyquist4}
\end{figure}

\begin{figure}[h!]
\center
\includegraphics[scale=0.25]{../code/figures/nyquist5.png}
\caption{Nyquist diagram for $K_{vs}=6.3$}
\label{fig:nyquist5}
\end{figure}

\begin{figure}[h!]
\center
\includegraphics[scale=0.25]{../code/figures/nyquist6.png}
\caption{Nyquist diagram for $K_{vs}=10$}
\label{fig:nyquist6}
\end{figure}

\begin{figure}[h!]
\center
\includegraphics[scale=0.25]{../code/figures/nyquist7.png}
\caption{Nyquist diagram for $K_{vs}=16$}
\label{fig:nyquist7}
\end{figure}

\subsection{Describing Function Results}

From the relationship given in (\ref{equ:selfOsc}) and the frequency of oscillation provided by (\ref{equ:systemTFiw}) and (\ref{equ:systemTFreal}), we find the frequency of oscillation to be:

\begin{equation}
w=2
\label{equ:omega_zero}
\end{equation}

At this frequency, equation (\ref{equ:maximum_k}) provides a maximum value of $K$ value of:

\begin{equation}
K_{MAX} < 5.10
\label{equ:k_valid}
\end{equation}

Since the valves come in specific sizes (1, 1.6, 2.5, 4.0, 6.3, 10, and 16) we find that the maximum valve coefficient is:

\begin{equation}
K_{vs} = 4.0
\label{equ:k_max}
\end{equation}


\section{Discussion}
\section{Conclusion}




\clearpage
\bibliography{reference}

\clearpage
\appendix

\section{Appendix}
Here is the Matlab code used to generate the results in the report and the figures.

\lstinputlisting{../code/main.m}

\end{document}